%!/usr/bin/env pdflatex
%-*- coding: utf-8 -*-
%@author : Romain Graux
%@date : 2022 April 24, 11:50:32
%@last modified : 2022 June 17, 18:46:23

Data is an important part of the society nowadays, it is responsible for the growth of the economy, the growth of the society, the growth of the world. 
The rise of what is known as Big Data will Facilitate things like newscasting (real-time forecasting of events), the development of inferential software that assesses project outcomes to date patterns, and the creation of advanced algorithms for correlations that enable a new understanding of the world.
Overall, the rise of big data is hugely positive for society in all aspects. The other question presented negative bias, including the energy consumption to produce, process and store them.
The Independent reported in 2016 that data centers will consume three times as much energy as they are currently using over the next decade. \cite{bib:data_storage_impact}
Furthemore, the data production is growing exponentially while the storage hardware is created from finite resources.

It becomes naturally important to find more eco-friendly and capable ways to treat these data. 
In this field, a new paradigm has emerged: DNA-based storage.
The genetic material DNA has garnered considerable interest as a medium for digital information storage because its density, durability and energy efficiency are superior to those of existing silicon-based storage media.  
Despite these advantages, DNA has not yet become a widespread information storage medium because the cost and the complexity of chemically synthesizing DNA are still prohibitively high. 
That is why researchers are still working on the development of a cheaper and more efficient DNA-based storage medium.

In this new era, it is important to explore new ways to store all types of information. 
It is therefore interesting to create new algorithms that are specifically made to encode any type of information directly into an ACGT code. 

With the rising of autonomous vehicles based on Lidar sensors and the integration of Lidar cameras into smartphones, more and more $3$ dimensional data is being produced which was not the case before. These data are rather heavy to store in their raw format.
They are therefore a good example of data that should be stored in a new type of medium that is more efficient. 

In this document we will explore how it would be possible to mix these two ideas to produce a single algorithm capable of storing these point clouds into a DNA-based medium.
